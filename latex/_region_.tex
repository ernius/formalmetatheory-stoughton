\message{ !name(diapositivas.tex)}\documentclass[utf,utf8x,hyperref=hidelinks]{beamer}

% no margen
%\usepackage[margin=0.1in]{geometry}
\usepackage{graphicx}
%\usepackage[bw,references]{latex/agda}
%\usepackage[conor,references]{latex/agda}
\usepackage[references,links]{agda}
\usepackage{amsmath}
\usepackage{mathtools}
\usepackage{textgreek}
\usepackage{catchfilebetweentags}
\usepackage{tipa}

%math
\newcommand{\alp}{\ensuremath{\alpha}}
\newcommand{\lamb}{\ensuremath{\lambda}}
\newcommand{\alphaeqsym}{\ensuremath{\sim_\alpha}}
\newcommand{\choice}{\ensuremath{\chi}}

%Agda
\newcommand{\freshin}[2]{\ensuremath{#1 \mathbin{\AgdaDatatype{\#}} #2}}
\newcommand{\lambAg}[2]{\ensuremath{\AgdaInductiveConstructor{ƛ}\, #1\, #2}}
\newcommand{\inAg}{\ensuremath{\mathbin{\AgdaFunction{∈}}}}
\newcommand{\ninAg}{\ensuremath{\mathbin{\AgdaFunction{∉}}}}
\newcommand{\neqAg}{\ensuremath{\mathbin{\AgdaInductiveConstructor{≢}}}}
\newcommand{\ap}[2]{#1 \ensuremath{\mathbin{\AgdaInductiveConstructorFunction{·}} #2}}
\newcommand{\var}[1]{\ensuremath{\AgdaInductiveConstructorFunction{v}\, #1}}
\newcommand{\fv}{\ensuremath{{\AgdaFunction{fv}}\,}}
\newcommand{\perm}{\ensuremath{\mathbin{\AgdaFunction{∙}}}}
\newcommand{\perma}{\ensuremath{\mathbin{\AgdaFunction{∙}_a}}}
\newcommand{\free}{\ensuremath{\mathbin{\AgdaFunction{*}}}}
\newcommand{\choiceAg}{\ensuremath{\AgdaFunction{χ}\,}}
\newcommand{\choiceAgaux}{\ensuremath{\AgdaFunction{χ'}\,}}
\newcommand{\alpeqAg}{\ensuremath{\mathbin{\AgdaDatatype{∼α}}}}
\newcommand{\swap}[3]{\ensuremath{(#1 \mathbin{\AgdaFunction{∙}} #2)\, #3}}

% \newcommand{\agdaf}[1]{\ensuremath{\AgdaFunction{#1}\,}}
% \newcommand{\agdaD}[1]{\ensuremath{\AgdaDatatype{#1}\,}}
% \newcommand{\agdav}[1]{\ensuremath{\AgdaBound{#1}\,}}

\DeclareUnicodeCharacter{411}{\textipa{\textcrlambda}}
\DeclareUnicodeCharacter{65288}{(}
\DeclareUnicodeCharacter{65289}{)}
\DeclareUnicodeCharacter{8788}{\ensuremath{\coloneqq}}
\DeclareUnicodeCharacter{8336}{\ensuremath{_a}}
\DeclareUnicodeCharacter{8799}{\ensuremath{\overset{?}{=}}}
\DeclareUnicodeCharacter{8759}{\ensuremath{\dblcolon}}
\DeclareUnicodeCharacter{8718}{\ensuremath{\square}}
\DeclareUnicodeCharacter{9657}{\ensuremath{\triangleright}}

%\usetheme{Berlin}
%\inserttocsection
\mode<presentation>
%% \newcommand\Section[2][]{%
%%    \section<presentation>[#1]{#2}
%%    \section<article>{#2}
%% }

\begin{document}

\message{ !name(diapositivas.tex) !offset(-3) }


%% \AgdaTarget{Λ}
%% \ExecuteMetaData[Term.tex]{term} 
\title{Alpha-Structural Induction and Recursion for the Lambda Calculus in Constructive Type Theory}
\author{Ernesto Copello,\'Alvaro Tasistro,Nora Szasz, Ana Bove, Maribel Fern\'andez}

\date[LSFA ’15]{10th Workshop on Logical and Semantic Frameworks, with Applications.}

\begin{frame}
\titlepage
\end{frame}

\begin{frame}
\frametitle{Outline}
\tableofcontents
\end{frame}

\section{Motivation}

\begin{frame}

\begin{block}{Motivation}
  We are interested in studying and formalising rasoning techniques over programming languages.
  \begin{itemize}
  \item like pen-and-paper ones
  \item using constructive type theory as proof assistant
  \end{itemize}

  \smallskip

  As $\lambda$-calculus is a minimal one, we want to formilise its properties in Agda.
\end{block}

\pause

\begin{table}
\begin{tabular}{c | c | c | }
Strategy & Pros & Cons \\
\hline 
John T & 13:04 & 24:15  \onslide<2-> \\ 
Norman P & 8:00 & 22:45 \onslide<3->\\
Alex K & 14:00 & 28:00  \onslide<4->\\
Sarah H & 9:22 & 21:10 
\end{tabular}
\end{table}

% \begin{block}{Several known formalisation strategies}
%   \begin{minipage}{0.3\linewidth}
%     \begin{itemize}
%     \item<1-> One sort variables 
%     \item<3-> Frege 
%     \item<5-> De Bruijn 
%     \item<7-> Locally Nameless
%     \item<9-> Hoas, Weak-Hoas
%     \item<11-> Nominal
%     \end{itemize}
%   \end{minipage}
%   \begin{minipage}{0.68\linewidth}
%     \begin{itemize}
%     \item<2-> capture avoiding substitution, reasoning over $\alpha$-equivalence classes \emph{BVC}
%     \item<4-> reasoning over $\alpha$-equivalence classes
%     \item<6-> introduces indexes operations, well-formed predicates, distance from intuitive classical proofs
%     \item<8-> need less indexes operations
%     \item<10-> has limitations
%     \item<12-> we study this strategy in this work
%     \end{itemize}
%   \end{minipage}
\end{block}

% \pause
% \begin{block}{Which to choose ?}
% \end{block}


\end{frame}

\end{document}

\message{ !name(diapositivas.tex) !offset(-142) }
