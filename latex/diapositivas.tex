\documentclass[utf,utf8x,hyperref=hidelinks,xcolor=table]{beamer}

% no margen
%\usepackage[margin=0.1in]{geometry}
\usepackage{graphicx}
%\usepackage[bw,references]{latex/agda}
%\usepackage[conor,references]{latex/agda}
\usepackage[references,links]{agda}
\usepackage{amsmath}
\usepackage{mathtools}
\usepackage{textgreek}
\usepackage{catchfilebetweentags}
\usepackage{tipa}

%math
\newcommand{\alp}{\ensuremath{\alpha}}
\newcommand{\lamb}{\ensuremath{\lambda}}
\newcommand{\alphaeqsym}{\ensuremath{\sim_\alpha}}
\newcommand{\choice}{\ensuremath{\chi}}

%Agda
\newcommand{\freshin}[2]{\ensuremath{#1 \mathbin{\AgdaDatatype{\#}} #2}}
\newcommand{\lambAg}[2]{\ensuremath{\AgdaInductiveConstructor{ƛ}\, #1\, #2}}
\newcommand{\inAg}{\ensuremath{\mathbin{\AgdaFunction{∈}}}}
\newcommand{\ninAg}{\ensuremath{\mathbin{\AgdaFunction{∉}}}}
\newcommand{\neqAg}{\ensuremath{\mathbin{\AgdaInductiveConstructor{≢}}}}
\newcommand{\ap}[2]{#1 \ensuremath{\mathbin{\AgdaInductiveConstructorFunction{·}} #2}}
\newcommand{\var}[1]{\ensuremath{\AgdaInductiveConstructorFunction{v}\, #1}}
\newcommand{\fv}{\ensuremath{{\AgdaFunction{fv}}\,}}
\newcommand{\perm}{\ensuremath{\mathbin{\AgdaFunction{∙}}}}
\newcommand{\perma}{\ensuremath{\mathbin{\AgdaFunction{∙}_a}}}
\newcommand{\free}{\ensuremath{\mathbin{\AgdaFunction{*}}}}
\newcommand{\choiceAg}{\ensuremath{\AgdaFunction{χ}\,}}
\newcommand{\choiceAgaux}{\ensuremath{\AgdaFunction{χ'}\,}}
\newcommand{\alpeqAg}{\ensuremath{\mathbin{\AgdaDatatype{∼α}}}}
\newcommand{\swap}[3]{\ensuremath{(#1 \mathbin{\AgdaFunction{∙}} #2)\, #3}}

% \newcommand{\agdaf}[1]{\ensuremath{\AgdaFunction{#1}\,}}
% \newcommand{\agdaD}[1]{\ensuremath{\AgdaDatatype{#1}\,}}
% \newcommand{\agdav}[1]{\ensuremath{\AgdaBound{#1}\,}}

\DeclareUnicodeCharacter{411}{\textipa{\textcrlambda}}
\DeclareUnicodeCharacter{65288}{(}
\DeclareUnicodeCharacter{65289}{)}
\DeclareUnicodeCharacter{8788}{\ensuremath{\coloneqq}}
\DeclareUnicodeCharacter{8336}{\ensuremath{_a}}
\DeclareUnicodeCharacter{8799}{\ensuremath{\overset{?}{=}}}
\DeclareUnicodeCharacter{8759}{\ensuremath{\dblcolon}}
\DeclareUnicodeCharacter{8718}{\ensuremath{\square}}
\DeclareUnicodeCharacter{9657}{\ensuremath{\triangleright}}

%\usetheme{Berlin}
%\inserttocsection
\mode<presentation>
%% \newcommand\Section[2][]{%
%%    \section<presentation>[#1]{#2}
%%    \section<article>{#2}
%% }

\addtobeamertemplate{frametitle}{\vspace*{-0.1cm}}{\vspace*{-0.5cm}}

\begin{document}

%% \AgdaTarget{Λ}
%% \ExecuteMetaData[Term.tex]{term} 
\title{Alpha-Structural Induction and Recursion for the Lambda Calculus in Constructive Type Theory}
\author{Ernesto Copello,\'Alvaro Tasistro,Nora Szasz, Ana Bove, Maribel Fern\'andez}

\date[LSFA ’15]{10th Workshop on Logical and Semantic Frameworks, with Applications.}

\begin{frame}
\titlepage
\end{frame}

\begin{frame}
\frametitle{Outline}
\tableofcontents
\end{frame}

\section{Motivation}

\begin{frame}{Motivation}
\begin{block}{}
  Studying and formalising reasoning techniques over programming languages.
  \begin{itemize}
  \item like pen-and-paper ones
  \item using constructive type theory as proof assistant
  \end{itemize}

  More specifically: $\lambda$-calculus \& Agda.
\end{block}

\pause
  \smallskip
\footnotesize{
  \begin{tabular}{c|c|c}
    Approach & Pros & Cons \\
    \hline
    One sort variables  & Simple one &
                          \begin{tabular}{c}
                            capture avoiding substitution \\
                            (Church 36 / Curry-Feys 58) \\
                            reasoning over $\alpha$-eq. classes (\emph{BVC}) \\
                          \end{tabular}  \onslide<3-> \\ 
    \hline

    Frege     & substitution & reasoning over $\alpha$-eq. classes \onslide<4->\\
    \hline

    De Bruijn & $\alpha$-eq. classes & 
                          \begin{tabular}{rl}
                               & indexes operations \\
                             + & well-formed predicates \\
                            \hline
                               & distance from intuitive proofs
                           \end{tabular}  \onslide<5->\\
    \hline

    Locally Nameless & less indexes operations & distance from intuitive proofs   \onslide<6->\\
    \hline

    HOAS      & substitution & has limitations   \onslide<7->\\
    \hline

    Nominal   & 
                \begin{tabular}{c}
                  swapping \\
                  \cellcolor{blue!25} $\alpha$-eq. classes 
                \end{tabular} 
             &
                \begin{tabular}{c}
               introduce swapping lemmas \\
               choice axiom incompatible
               \end{tabular}
  \end{tabular}}
\end{frame}

\section{Reasoning over $\alpha$-equivalence classes}

\begin{frame}{Reasoning over $\alpha$-equivalence classes}

  \begin{block}{Barenregt's variable convention}
    Each $\lambda$-term represents its $\alpha$ class, so it could be assumed to have bound and context variables all different.
  \end{block}

  \medskip
 \pause

  \begin{block}{Formilising $\alpha$-structural induction principle in a classic approach}
    \begin{itemize}
    \item Complete induction over the \emph{size} of terms is needed to fill the gap between terms and $\alpha$-equivalence classes.
    \item Prove that the property being proved is $\alpha$-equivalent.
    \end{itemize}
  \end{block}

  \smallskip
 \pause

  \begin{block}{Complete induction scketch}
     \underline{$\lambda$\ case:} To prove  $\forall x, M,\ P(\lambda x M)$\ we can instead prove:
             \[ \exists x^*, M^*\ \text{renamings} / \lambda x M \alphaeqsym \lambda x^* M^*,\ P(M^*) \Rightarrow P(\lambda x^* M^*)\]
     
     
  \end{block}

\end{frame}


\section{Nominal Abstract Syntax}

\end{document}
