\documentclass{article} 

% no margen
%\usepackage[margin=0.1in]{geometry}

\usepackage{graphicx}

%\usepackage[bw,references]{latex/agda}
%\usepackage[conor,references]{latex/agda}
\usepackage[hidelinks]{hyperref}
\usepackage[references,links]{agda}
\usepackage{amsmath}
\usepackage{amsthm}
\usepackage{mathtools}
\usepackage{textgreek}
\usepackage{catchfilebetweentags}
\usepackage{tipa}
\usepackage{bussproofs}
\usepackage{color}

%math
\newcommand{\noalp}{\ensuremath{no\alpha}}
\newcommand{\p}{\ensuremath{\rightrightarrows}}
\newcommand{\betaalpha}{\ensuremath{\rightarrow_\alpha}}
\newcommand{\betaaster}{\ensuremath{\rightarrow_\beta^*}}
\newcommand{\alp}{\ensuremath{\alpha}}
\newcommand{\alpsy}{\ensuremath{\sim}}
\newcommand{\lam}{\ensuremath{\lambda}}
\newcommand{\alphaeqsym}{\ensuremath{\sim_\alpha}}
\newcommand{\choice}{\ensuremath{\chi}}
\newcommand{\add}[2]{\ensuremath{\prec\!\!+(#1,#2)}}

%Agda
\newcommand{\freshin}[2]{\ensuremath{#1 \mathbin{\AgdaDatatype{\#}} #2}}
\newcommand{\lambAg}[2]{\ensuremath{\AgdaInductiveConstructor{ƛ}\, #1\, #2}}
\newcommand{\inAg}{\ensuremath{\mathbin{\AgdaFunction{∈}}}}
\newcommand{\ninAg}{\ensuremath{\mathbin{\AgdaFunction{∉}}}}
\newcommand{\neqAg}{\ensuremath{\mathbin{\AgdaInductiveConstructor{≢}}}}
\newcommand{\ap}[2]{#1 \ensuremath{\mathbin{\AgdaInductiveConstructorFunction{·}} #2}}
\newcommand{\var}[1]{\ensuremath{\AgdaInductiveConstructorFunction{v}\, #1}}
\newcommand{\fv}{\ensuremath{{\AgdaFunction{fv}}\,}}
\newcommand{\perm}{\ensuremath{\mathbin{\AgdaFunction{∙}}}}
\newcommand{\perma}{\ensuremath{\mathbin{\AgdaFunction{∙}_a}}}
\newcommand{\free}{\ensuremath{\mathbin{\AgdaFunction{*}}}}
\newcommand{\choiceAg}{\ensuremath{\AgdaFunction{χ}\,}}
\newcommand{\choiceAgaux}{\ensuremath{\AgdaFunction{χ'}\,}}
\newcommand{\alpeqAg}{\ensuremath{\mathbin{\AgdaDatatype{∼α}}}}
\newcommand{\swap}[3]{\ensuremath{(#1 \mathbin{\AgdaFunction{∙}} #2)\, #3}}

% \newcommand{\agdaf}[1]{\ensuremath{\AgdaFunction{#1}\,}}
% \newcommand{\agdaD}[1]{\ensuremath{\AgdaDatatype{#1}\,}}
% \newcommand{\agdav}[1]{\ensuremath{\AgdaBound{#1}\,}}

\DeclareUnicodeCharacter{411}{\textipa{\textcrlambda}}
\DeclareUnicodeCharacter{65288}{(}
\DeclareUnicodeCharacter{65289}{)}
\DeclareUnicodeCharacter{8788}{\ensuremath{\coloneqq}}
\DeclareUnicodeCharacter{8336}{\ensuremath{_a}}
\DeclareUnicodeCharacter{8799}{\ensuremath{\overset{?}{=}}}
\DeclareUnicodeCharacter{8759}{\ensuremath{\dblcolon}}
\DeclareUnicodeCharacter{8718}{\ensuremath{\square}}
\DeclareUnicodeCharacter{9657}{\ensuremath{\triangleright}}
\DeclareUnicodeCharacter{8649}{\ensuremath{\rightrightarrows}}
\DeclareUnicodeCharacter{8339}{}
\DeclareUnicodeCharacter{8347}{}
\DeclareUnicodeCharacter{9635}{\ensuremath{\qed}}

\newtheorem{lem}{Lemma}

\begin{document}

Parallel Reduction of Stoughton's paper:

\ExecuteMetaData[ParallelReduction.tex]{parallel} 

We already have the following result in the paper:

\ExecuteMetaData[SubstitutionLemmas.tex]{parallelsubstitutionlemma}

The alpha rule of this parallel relation definition brings high non determinism in the construction evidence. In fact, infinite of them exists for any two related lambda terms.
So we define a predicate that holds only if a parallel reduction has no alpha rules in it.

\ExecuteMetaData[ChurchRosser.tex]{noalphaparallel} 

Next, we prove that given a parallel reduction $M \p N$\ then it exists a term $P$, alpha equivalent to $N$, such that $M \p P$ with no alpha rules in the derivations. 

\ExecuteMetaData[ChurchRosser.tex]{noalphaparallellemma} 

\begin{proof} By induction on parallel relation.
\begin{itemize}
   \item[var. rule:] The proof is direct.
   \item[abs. rule:]       
\begin{minipage}{0.2\linewidth}
     Hypotheses:
\end{minipage}
\begin{minipage}{0.4\linewidth}
\AxiomC{$M  \p N$}
\LeftLabel{$\lam$}
\UnaryInfC{$\lam x M \p \lam x N$}
\DisplayProof
\end{minipage}

Thesis: $\exists P, \lam x N \alpsy P, \lam x M \p P$ with no \alp\ rules.

Proof:

\AxiomC{$M  \p N$}
\LeftLabel{ind. hyp}
\UnaryInfC{$\exists P', P' \alpsy N, M \p P'$ with no \alp\ rules}
\LeftLabel{\p\lam}
\UnaryInfC{$\lam x M \p \lam x P'$ with no \alp\ rules}
\AxiomC{$P'  \alpsy N$}
\UnaryInfC{$\lam x P' \alpsy \lam x N$}
\BinaryInfC{$\exists P = \lam x P', \lam x N \alpsy P, \lam x M \p P$ with no \alp\ rules}
\DisplayProof

   \item[app. rule:]       
\begin{minipage}{0.2\linewidth}
     Hypotheses:
\end{minipage}
\begin{minipage}{0.4\linewidth}
\AxiomC{$M  \p M'$}
\AxiomC{$N  \p N'$}
\LeftLabel{\p app}
\BinaryInfC{$M N \p M' N'$}
\DisplayProof
\end{minipage}

Thesis: $\exists P, M N \alpsy P, M N \p P$ with no \alp\ rules.

Proof:
\AxiomC{$M  \p M'$}
\LeftLabel{ind. hyp}
\UnaryInfC{$\exists Q, Q \alpsy M', M \p Q$ with no \alp\ rules}
\DisplayProof

\AxiomC{$N  \p N'$}
\LeftLabel{ind. hyp}
\UnaryInfC{$\exists R, R \alpsy N', M \p R$ with no \alp\ rules}
\DisplayProof

Then exists $P = Q R$ such that $M' N' \alpsy P$ by alpha application rule, and $M N \p P$ by parallel application rules with no \alp\ rules in the derivation.

   \item[$\beta$ rule:]       
\begin{minipage}{0.2\linewidth}
     Hypotheses:
\end{minipage}
\begin{minipage}{0.4\linewidth}
\AxiomC{$M  \p M'$}
\AxiomC{$N  \p N'$}
\LeftLabel{$\p \beta$}
\BinaryInfC{$\lam x M N \p M' [x := N']$}
\DisplayProof
\end{minipage}

Thesis: $\exists P, M' [x := N'] \alpsy P, \lam x M N \p P$ with no \alp\ rules.

Proof:
\AxiomC{$M  \p M'$}
\LeftLabel{ind. hyp}
\UnaryInfC{$\exists Q, Q \alpsy M', M \p Q$ with no \alp\ rules}
\DisplayProof

\AxiomC{$N  \p N'$}
\LeftLabel{ind. hyp}
\UnaryInfC{$\exists R, R \alpsy N', N \p R$ with no \alp\ rules}
\DisplayProof

\begin{minipage}{0.1\linewidth}
(1)
\end{minipage}
\begin{minipage}{0.9\linewidth}
\AxiomC{$M'  \alpsy Q$}
\AxiomC{$N'  \alpsy R$}
\LeftLabel{subst. lemma}
\BinaryInfC{$M' [x := N'] \alpsy Q [x := R']$}
\DisplayProof
\end{minipage}

\begin{minipage}{0.1\linewidth}
(2)
\end{minipage}
\begin{minipage}{0.9\linewidth}
\AxiomC{$M  \p Q$ no \alp\ rules}
\AxiomC{$N  \p R$ no \alp\ rules}
\LeftLabel{$\p\beta$}
\BinaryInfC{$\lam x M N \alpsy Q [x := R]$ no \alp\ rules}
\DisplayProof
\end{minipage}

Then exists $P = Q [x := R]$\ such that $M' [x := N'] \alpsy P$ by (1)$, \lam x M N \p P$ with no \alp\ rules by (2).

   \item[\alp\ rule:]       
\begin{minipage}{0.2\linewidth}
     Hypotheses:
\end{minipage}
\begin{minipage}{0.4\linewidth}
\AxiomC{$M  \p N$}
\AxiomC{$N  \alpsy Q$}
\LeftLabel{$\p\alpha$}
\BinaryInfC{$M \p Q$}
\DisplayProof
\end{minipage}

Thesis: $\exists P, Q \alpsy P, M \p P$ with no \alp\ rules.

Proof:
\AxiomC{$M  \p N$}
\LeftLabel{ind. hyp}
\UnaryInfC{$\exists R, R \alpsy N, M \p R$ with no \alp\ rules}
\UnaryInfC{$\exists P = R/$ by transitive of \alpsy\ $Q \alpsy P, M \p P$ with no \alp\ rules}
\DisplayProof


\end{itemize}
\end{proof}

With previous result we are able to prove that the parallel relation is also alpha compatible in the left term of the relation.

\ExecuteMetaData[ChurchRosser.tex]{parallelleftalpha}

{\color{red} Nuevo: This result can not be proved by induction on the term $M$, neither its length, because in the alpha case of the parallel relation the term in the premises does not decrease. The only remaining choice is an induction on the parallel relation, which fails due to our substitution operation always renames abstraction variables. We show the failing abstraction case to make explicit this problematic behaviour.

\begin{itemize}
   \item[abs. case:] $M = \lam x M$

\begin{minipage}{0.2\linewidth}
           Hypotheses: 
\end{minipage}
\begin{minipage}{0.2\linewidth}
$\lam x M \alpsy \lam y N $ , 
\end{minipage}
\begin{minipage}{0.6\linewidth}
\AxiomC{$N \p P$}
\LeftLabel{$\p abs.$}
\UnaryInfC{$\lam y N \p \lam y P$}
\DisplayProof          
\end{minipage}

Tesis: $\lam x M \p \lam y P$

Proof: 

By being doing the induction over the parallel relation, we have to apply the inductive hypothesis to $N \p P$.  Then we proceed as follows:

\AxiomC{$\lam x M  \alpsy \lam y N$}
\UnaryInfC{$M [x := y] \alpsy N$}

\AxiomC{$N \p P$}
\LeftLabel{h.i.}
\BinaryInfC{$M [x := y] \p P$}
\LeftLabel{subst. lemma \p}
\UnaryInfC{$M [x := y] [y := x]\p P [y := x] $}
\DisplayProof

Here the problem comes because of $M [x := y] [y := x]$\ is not equals to $M$, as our substitution normilises terms, so we only have that $M [x := y] [y := x] \alpsy M$. Here appears the need of an induction on the term to be able to apply inductive hypothesis again and obtain that $M \p P [y := x]$, and then continue as follows:

\AxiomC{$M \p P [y := x]$}
\LeftLabel{\p\lam}
\UnaryInfC{$\lam x M \p \lam x P [y := x]$}
\AxiomC{$\lam x P [y := x] \alpsy \lam y P$}
\LeftLabel{\p\alp}
\BinaryInfC{$\lam x M \p \lam y P $}
\DisplayProof

But as we already explained due to alpha rule it is imposible an inductions on terms. So we introduce $\p_0$ relation to pospone the alpha problematic rule, and be able to do an induction on terms.
% \LeftLabel{subst. lemma \p}
% \UnaryInfC{$N[y:=x] \p P[y:=x]$}
% \LeftLabel{lemma\p no$\alpha$}
% \UnaryInfC{$\exists Q, Q \alpsy P[y:=x], N[y:=x]\p Q$ no \alp\ rules}
% \LeftLabel{ind. hyp.}
% \BinaryInfC{$M \p Q$}
% \DisplayProof


% \AxiomC{$M \p Q$}
% \AxiomC{$Q \alpsy P[y:=x]$}
% \LeftLabel{\p \alp}
% \BinaryInfC{$M \p P[y:=x]$}
% \LeftLabel{\p abs.}
% \UnaryInfC{$\lam x M  \p \lam x P[y:=x]$}

% \AxiomC{$x \# \lam x M$}
% \AxiomC{$\lam x M \alpsy \lam y N $}
% \BinaryInfC{$x \# \lam y N$}
% \AxiomC{$\lam y N \p \lam y P$}
% \BinaryInfC{$x \# \lam y P$}
% \UnaryInfC{$\lam x P[y:=x] \alpsy \lam y P$}
% \LeftLabel{\p \alp}
% \BinaryInfC{$\lam x M \p \lam y P$}
% \DisplayProof

\end{itemize}

 }

To prove this result we will use the next similar one, it adds the extra premise that the parallel relation hypothesis has no alpha rules in its derivation.

\ExecuteMetaData[ChurchRosser.tex]{parallellnoalphaeftalpha}

With this weakened lemma and the previously proved lemma\p no\alp\ we directly prove the desired result.

\begin{proof}
\AxiomC{$M  \alpsy N$}
\AxiomC{$N  \p P$}
\LeftLabel{lemma\p no\alp}
\UnaryInfC{$\exists Q, Q \alpsy P, N \p Q$ with no \alp rules}
\LeftLabel{prev. lemma}
\BinaryInfC{$M \p Q$}
\AxiomC{$Q  \alpsy P$}
\LeftLabel{\p \alp}
\BinaryInfC{$M \p P$}
\DisplayProof 
\end{proof}

Next, we prove our auxiliary lemma.

\begin{proof} By structural induction on the left term $M$ of the alpha-equivalence relation hypothesis

\begin{itemize}
   \item[var. case:] $M = x$\ this case is direct.
   \item[app. case:] $M = M M'$\ it has two subcases depending on the last rule on the parallel relation derivation, which could be an application or beta rule (alpha rule case is discared by the extra hypothesis).

     \begin{itemize}
     \item[app. rule:] 

         \begin{tabular}[t]{cc}
           Hypotheses: & 
\AxiomC{$M  \alpsy N$}
\AxiomC{$M'  \alpsy N'$}
\LeftLabel{$\alpsy\ app.$}
\BinaryInfC{$M M' \alpsy N N'$}
\DisplayProof          \\

                        & 
\AxiomC{$N  \p P$ with no \alp\ rules}
\AxiomC{$N'  \p P'$ with no \alp\ rules}
\LeftLabel{$\p\ app.$}
\BinaryInfC{$N N' \p P P'$}
\DisplayProof

                       \\
         \end{tabular}

Tesis: $M M' \p P P'$

Proof:
\AxiomC{$M  \alpsy N$}
\AxiomC{$N  \p P$ no \alp\ rules}
\LeftLabel{ind. hyp.}
\BinaryInfC{$M \p P$}

\AxiomC{$M'  \alpsy N'$}
\AxiomC{$N'  \p P'$ no \alp\ rules}
\LeftLabel{ind. hyp.}
\BinaryInfC{$M' \p P'$}

\LeftLabel{\p app.}
\BinaryInfC{$M M' \p P P'$}
\DisplayProof

     \item[$\beta$\ rule:]

         \begin{tabular}[t]{cc}
           Hypotheses: $M = \lam x M M'$& 

\AxiomC{$\lam x M  \alpsy \lam y N$}
\AxiomC{$M'  \alpsy N'$}
\LeftLabel{$\alpsy\ app.$}
\BinaryInfC{$\lam x M M' \alpsy \lam y N N'$}
\DisplayProof          \\

                        & 
\AxiomC{$ N  \p P$ with no \alp\ rules}
\AxiomC{$N'  \p P'$ with no \alp\ rules}
\LeftLabel{$\p\ app.$}
\BinaryInfC{$\lam y N N' \p P [y := P']$ }
\DisplayProof

                       \\
         \end{tabular}

Tesis: $\lam x M M' \p P [y := P']$

Proof: 

\AxiomC{$\lam x M  \alpsy \lam y N$}
\UnaryInfC{$M \alpsy N[y:=x]$}
\AxiomC{$N \p P$}
\LeftLabel{subst. lemma \p}
\UnaryInfC{$N[y:=x] \p P[y:=x]$}
\LeftLabel{lemma\p no$\alpha$}
\UnaryInfC{$\exists Q, Q \alpsy P[y:=x], N[y:=x]\p Q$ no \alp\ rules}
\LeftLabel{ind. hyp.}
\BinaryInfC{$M \p Q$}
\DisplayProof

\AxiomC{$M \p Q$}
\AxiomC{$Q \alpsy P[y:=x]$}
\LeftLabel{\p \alp}
\BinaryInfC{$M \p P[y:=x]$}

\AxiomC{$M' \alpsy N'$}
\AxiomC{$N' \p P'$}
\LeftLabel{ind. hyp.}
\BinaryInfC{$M' \p P'$}

\LeftLabel{\p $\beta$}
\BinaryInfC{$\lam x M M' \p P[y:=x][x:=P']$}
\DisplayProof


\AxiomC{$\lam x M M' \p P[y:=x][x:=P']$}

\AxiomC{$x \# \lam x M$}
\AxiomC{$\lam x M \alpsy \lam y N $}
\BinaryInfC{$x \# \lam y N$}
\AxiomC{$\lam y N \p \lam y P$}
\BinaryInfC{$x \# \lam y P$}
\UnaryInfC{$P[y:=x][x:=P'] \alpsy P[y:=P']$}
\LeftLabel{\p \alp}
\BinaryInfC{$\lam x M M' \p P[y:=P']$}
\DisplayProof

     \end{itemize}

   \item[abs. case:] $M = \lam x M$

\begin{minipage}{0.2\linewidth}
           Hypotheses: 
\end{minipage}
\begin{minipage}{0.2\linewidth}
$\lam x M \alpsy \lam y N $ , 
\end{minipage}
\begin{minipage}{0.6\linewidth}
\AxiomC{$N \p P$}
\LeftLabel{$\p abs.$}
\UnaryInfC{$\lam y N \p \lam y P$}
\DisplayProof          
\end{minipage}

Tesis: $\lam x M \p \lam y P$

Proof: 

\AxiomC{$\lam x M  \alpsy \lam y N$}
\UnaryInfC{$M \alpsy N[y:=x]$}

\AxiomC{$N \p P$}
\LeftLabel{subst. lemma \p}
\UnaryInfC{$N[y:=x] \p P[y:=x]$}
\LeftLabel{lemma\p no$\alpha$}
\UnaryInfC{$\exists Q, Q \alpsy P[y:=x], N[y:=x]\p Q$ no \alp\ rules}
\LeftLabel{ind. hyp.}
\BinaryInfC{$M \p Q$}
\DisplayProof


\AxiomC{$M \p Q$}
\AxiomC{$Q \alpsy P[y:=x]$}
\LeftLabel{\p \alp}
\BinaryInfC{$M \p P[y:=x]$}
\LeftLabel{\p abs.}
\UnaryInfC{$\lam x M  \p \lam x P[y:=x]$}

\AxiomC{$x \# \lam x M$}
\AxiomC{$\lam x M \alpsy \lam y N $}
\BinaryInfC{$x \# \lam y N$}
\AxiomC{$\lam y N \p \lam y P$}
\BinaryInfC{$x \# \lam y P$}
\UnaryInfC{$\lam x P[y:=x] \alpsy \lam y P$}
\LeftLabel{\p \alp}
\BinaryInfC{$\lam x M \p \lam y P$}
\DisplayProof

\end{itemize}
\end{proof}

Finally, we present the diamond property of the parallel reduction.

\ExecuteMetaData[ChurchRosser.tex]{paralleldiamond}

To prove this result we use a similar approach to previous lemma, proving first a weakened version of it, which adds that the parallel reduction premises have no \alp\ rules in its derivations.

\ExecuteMetaData[ChurchRosser.tex]{paralleldiamondnoalpha}

With previous lemma  and the lemma\p no\alp result the proof is direct.

\begin{proof}
\AxiomC{$M \p N$}
\LeftLabel{lemma\p no\alp}
\UnaryInfC{$\exists N', N' \alpsy N, M \p N'$ with no \alp\ rules}
\DisplayProof

\AxiomC{$M \p P$}
\LeftLabel{lemma\p no\alp}
\UnaryInfC{$\exists P', P' \alpsy P, M \p P'$ with no \alp\ rules}
\DisplayProof

\AxiomC{$M \p N'$ with no \alp\ rules}
\AxiomC{$M \p P'$ with no \alp\ rules}
\LeftLabel{diam\p no\alp}
\BinaryInfC{$\exists Q, N'\p Q P'\p Q$}
\DisplayProof

\AxiomC{$N \alpsy N'$}
\AxiomC{$N' \p Q$}
\LeftLabel{lemma\p\alp left}
\BinaryInfC{$N \p Q$}

\AxiomC{$P \alpsy P'$}
\AxiomC{$P' \p Q$}
\LeftLabel{lemma\p\alp left}
\BinaryInfC{$P \p Q$}

\BinaryInfC{$\exists Q, N \p Q, P \p Q$}
\DisplayProof
\end{proof}

Next, we prove the auxiliary lemma.

\begin{proof}
By induction on the parallel reduction relations. Since \alp\ rules can not ocurr by hypotheses, the possible cases are reduced to the following ones:

\begin{itemize}
\item[var-var rules:] The proof is direct.
\item[abs-abs rules:] 
  \begin{tabular}[t]{cc}
   Hypotheses: &
\AxiomC{$M \p N$ with no \alp\ rules}
\LeftLabel{\p abs}
\UnaryInfC{$\lam x M \p \lam x N$}
\DisplayProof
               \\
               & 
\AxiomC{$M \p P$ with no \alp\ rules}
\LeftLabel{\p abs}
\UnaryInfC{$\lam x M \p \lam x P$}
\DisplayProof
  \end{tabular}


Thesis: $\exists Q, \lam x N \p Q, \lam x P \p Q$
  
Proof:

\AxiomC{$M \p N$ with no \alp\ rules}
\AxiomC{$M \p P$ with no \alp\ rules}
\LeftLabel{ind. hyp.}
\BinaryInfC{$\exists Q', N \p Q', P \p Q'$}
\DisplayProof

\AxiomC{$N \p Q'$}
\LeftLabel{\p abs}
\UnaryInfC{$\lam x N \p \lam x Q'$}

\AxiomC{$P \p Q'$}
\LeftLabel{\p abs}
\UnaryInfC{$\lam x P \p \lam x Q'$}

\BinaryInfC{$\exists Q = \lam x Q', \lam x N \p Q , \lam x P \p Q$}
\DisplayProof

\item[app-app rules:] 
  \begin{tabular}[t]{cc}
   Hypotheses: &
\AxiomC{$M \p N$ with no \alp\ rules}
\AxiomC{$M' \p N'$ with no \alp\ rules}
\LeftLabel{\p app}
\BinaryInfC{$M M' \p N N'$}
\DisplayProof
               \\
               & 
\AxiomC{$M \p P$ with no \alp\ rules}
\AxiomC{$M' \p P'$ with no \alp\ rules}
\LeftLabel{\p app}
\BinaryInfC{$M M' \p P P'$}
\DisplayProof
  \end{tabular}

Thesis: $\exists Q, N N' \p Q, P P' \p Q$
  
Proof:

\AxiomC{$M \p N$ with no \alp\ rules}
\AxiomC{$M \p P$ with no \alp\ rules}
\LeftLabel{ind. hyp.}
\BinaryInfC{$\exists Q', N \p Q', P \p Q'$}
\DisplayProof

\AxiomC{$M' \p N'$ with no \alp\ rules}
\AxiomC{$M' \p P'$ with no \alp\ rules}
\LeftLabel{ind. hyp.}
\BinaryInfC{$\exists Q'', N' \p Q'', P' \p Q''$}
\DisplayProof

\AxiomC{$N \p Q'$ }
\AxiomC{$N' \p Q''$}
\LeftLabel{\p app}
\BinaryInfC{$N N' \p Q' Q''$}

\AxiomC{$P \p Q'$ }
\AxiomC{$P' \p Q''$}
\LeftLabel{\p app}
\BinaryInfC{$P P' \p Q' Q''$}

\BinaryInfC{$\exists Q = Q' Q'', N N' \p Q, P P' \p Q$}
\DisplayProof


\item[beta-beta rules:] 
  \begin{tabular}[t]{cc}
   Hypotheses: &
\AxiomC{$M \p N$ with no \alp\ rules}
\AxiomC{$M' \p N'$ with no \alp\ rules}
\LeftLabel{$\p \beta$}
\BinaryInfC{$\lam x M M' \p N [ x := N']$}
\DisplayProof
               \\
               & 
\AxiomC{$M \p P$ with no \alp\ rules}
\AxiomC{$M' \p P'$ with no \alp\ rules}
\LeftLabel{$\p \beta$}
\BinaryInfC{$\lam x M M' \p P [ x := P' ]$}
\DisplayProof
  \end{tabular}

Thesis: $\exists Q, N [ x := N'] \p Q, P [x := P'] \p Q$
  
Proof:

\AxiomC{$M \p N$ with no \alp\ rules}
\AxiomC{$M \p P$ with no \alp\ rules}
\LeftLabel{ind. hyp.}
\BinaryInfC{$\exists Q', N \p Q', P \p Q'$}
\DisplayProof

\AxiomC{$M' \p N'$ with no \alp\ rules}
\AxiomC{$M' \p P'$ with no \alp\ rules}
\LeftLabel{ind. hyp.}
\BinaryInfC{$\exists Q'', N' \p Q'', P' \p Q''$}
\DisplayProof

Applying the substitution lemma of parallel reduction (lemma\p).

\AxiomC{$N \p Q'$}
\AxiomC{$N' \p Q''$}
\UnaryInfC{$[x := N'] \p [x := Q'']$}
\LeftLabel{lemma\p}
\BinaryInfC{$N [x := N'] \p Q' [ x := Q''] $}
\DisplayProof

In a similar way we have that $P [x := P'] \p Q' [ x := Q''] $, then exists $Q = Q' [ x := Q'']$ that satifies the thesis.

\item[beta-app rules:] Similar to the two previous cases combined.
\item[app-beta rules:] Simetric to previous case.

\end{itemize}
\end{proof}

Beta contraction

\ExecuteMetaData[Beta.tex]{betacontraction} 


Contextual closure of beta contraction.

\ExecuteMetaData[Beta.tex]{beta} 

Alpha beta union relation.

\ExecuteMetaData[Beta.tex]{alphabeta} 

Reflexive clausure of previous relation

\ExecuteMetaData[Beta.tex]{asteralphabeta} 

To end the proof of \betaalpha\ relation confluence we the following two results:

\begin{itemize}
\item $\betaalpha \subseteq \p$
\item $\p \subseteq \betaalpha^*$
\end{itemize}

Both proofs are a direct induction on the left relations on both inclusions. In the beta case of the second proof we use the following two subtitutions lemmas for the $\betaalpha^*$\ relation:

\ExecuteMetaData[Beta.tex]{betaalphasubstitutionterm} 

\begin{proof} 
This lemma is proved by induction on the reflexive and transitive closure of $\betaalpha$\ relation. The reflexive and transitive cases of the closure are direct. The only remaining case is the one step case, which can be an \alp\ or $\beta$\ step.

\begin{itemize}
 \item \alp\ case: As the substitution normilises \alp-convertible terms the result is a direct reflexive step.
 \item $\beta$ case: We do an induction on the closure cases, all are direct, except from the abstraction and $\beta$-contraction cases that we discuss next.

   \begin{itemize}
   \item abstraction case:

   \[ \begin{array}{rcll}
       (\lam x M) \sigma & \betaalpha^*\ & \{\text{by substitution definition, with}\ y = \choice\ \sigma\ (\lam x M) &\} \\
       \lam y (M \sigma \add{x}{y}) & \betaalpha^* & \{\text{by inductive hypotheses and that the} &\} \\
                                    &              &\{ \betaalpha^* \text{ relation is preserved by the application of an abstraction} &\} \\
 
       \lam y (M' \sigma \add{x}{y}) & \betaalpha^*& \{\text{by being \alp-convertible, with}\ y' = \choice\ \sigma\ (\lam x M') &\} \\
       \lam y' (M' \sigma \add{x}{y'}) & \betaalpha^*& \{\text{by substitution definition} &\} \\
       ((\lam x M') \sigma &&&
   \end{array} \]
   \item $\beta$\  case:

   \[ \begin{array}{rcll}
       ((\lam x M) N) \sigma & \betaalpha^*\ & \{\text{by substitution definition, with}\ y = \choice\ \sigma\ (\lam x M) &\} \\
       (\lam y (M \sigma \add{x}{y})) (N \sigma) & \betaalpha^* & \{\beta\text{-contraction} &\} \\
       (M \sigma \add{x}{y}) \iota \add{y}{N \sigma}  & \betaalpha^* & \{\text{substitution lemma corollary 1} &\} \\
       (M \sigma \add{x}{N \sigma}  & \betaalpha^* & \{\text{corollary 1 prop 7 Stoughton} &\} \\
       (M \iota \add{x}{N}) \sigma &&&
       \end{array} \]
  
   \end{itemize}
\end{itemize}
  
\end{proof}

Finally we prove the remaining lemma to complete the relations inclusions proof.

\ExecuteMetaData[Beta.tex]{alphabetaastersubstitutionsigma} 

\begin{proof} By induction on terms. The only non direct case is the following abstraction case:

   \[ \begin{array}{rcll}
       (\lam x M)  \sigma & \betaalpha^*\ & \{\text{by substitution definition, with}\ y = \choice\ \sigma\ (\lam x M) &\} \\
       \lam y (M \sigma \add{x}{y}))  & \betaalpha^* & \{\text{by inductive hypotheses and that the} &\} \\
                                    &              &\{ \betaalpha^* \text{ relation is preserved by the application of an abstraction} &\} \\
       \lam y (M \sigma' \add{x}{y})  & \betaalpha^* & \{\text{by being \alp-convertible, with}\ y' = \choice\ \sigma\ (\lam x M') &\} \\
       \lam y' (M \sigma' \add{x}{y'})  & \betaalpha^* & \{\text{by substitution definition} &\} \\
       (\lam x M) \sigma' &&&
       \end{array} \]

  
\end{proof}

\end{document}